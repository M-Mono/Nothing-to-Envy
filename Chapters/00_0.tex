\fancyhead[R]{{\scriptsize {\faBook\ }我们最幸福 > 作者的话}}
\chapter*{作者的话}
\addcontentsline{toc}{chapter}{\hspace{5mm}作者的话}
\vspace{5mm}
\begin{flushright}
	\textcolor{PinYinColor}{\EN \huge{Author's\\
			Note\\
			\ \\}}
\end{flushright}

2001年我被派往首尔,作为《洛杉矶时报》(Los Angeles Times)的特派记者,报导区域涵盖北朝鲜及南韩。在当时作为一个美国记者,访问北朝鲜是非常困难的。而且即使千方百计得以访问北朝鲜,我发现要完成一个报导也几乎是件不可能的事情。

访问北朝鲜的西方记者们通常都会被指派一个所谓的“看管”,他的工作就是确保不发生任何没有官方批准的交谈。同时,访问者所参观的地方都是事先经过精心挑选。同当地普通市民接触是绝对不允许的。在照片及电视里,有关北朝鲜人的形象,都是机器人似的、整齐划一的正步阅兵,就是出现在为歌颂领袖而举行的大型团体操中。我久久的凝视着这些照片,试图探究这些面无表情的面孔后面可能的故事。

在南韩我开始了与脱北者\footnote{从北朝鲜逃亡至南韩或者中国的朝鲜人。}进行交谈,一幅朝鲜民主主义人民共和国普通百姓真实生活的画卷慢慢展现在我眼前。我已经为《洛杉矶时报》(Los Angeles Times)写了一系列的文章,这些报导聚焦于来自北朝鲜最北部清津市的脱北者。我相信,针对来自同一地区的人员,交谈越多就越容易对一些事情进行相互印证。关于地区的选择,我倾向于选择那些远离北朝鲜政府朝所精心安排的,专门向外国访问者展示的地方,而这也就意味着我所要描写的地点对我来说是个禁地。清津市是北朝鲜第三大城市,而且也是受九十年代中期开始的饥荒波及最严重的地区之一。该地区迄今为止,仍然对外国人完全封闭。我十分有幸能遇到很多非常好的清津人,他们都知无不言、言无不尽也丝毫不吝啬自己的时间。《我们最幸福》(Nothing to Envy)这本书就来源于这一系列的报导。

本书是基于七年来对脱北者的访谈。出于保护那些至今仍然生活在北朝鲜的人们,书中我都采用了化名。所有的对话都是取自一名或多名当事人的描述。我也尽我所能将所听到的故事同公开报导的事件进行印证。书中对于我个人无法亲自参观地点的描述,来自于脱北者的口述、照片或者影像资料。北朝鲜在很多方面,迄今为止外界仍然不得而知。因此我也不能保证我所听到的都是事实真相。我所希望的就是,有朝一日北朝鲜变得开放之后,我们能够自己判断那里到底发生了什么。
