\fancyhead[LO]{{\scriptsize {\FA \ }我们最幸福 {\FA } 陷入黑暗}}%奇數頁眉的左邊
\fancyhead[RO]{{\tiny{\textcolor{Gray}{\FA \ }}}\thepage}
\fancyhead[LE]{{\tiny{\textcolor{Gray}{\FA \ }}}\thepage}
\fancyhead[RE]{{\scriptsize {\FA \ }我们最幸福 {\FA } 陷入黑暗}}%偶數頁眉的右邊
\fancyfoot[LE,RO]{}
\fancyfoot[LO,CE]{}
\fancyfoot[CO,RE]{}
\chapter*{04 {\FA } 陷入黑暗}
\addcontentsline{toc}{chapter}{\hspace{5mm}04 \textbf{>}\ \ 陷入黑暗}
\vspace{5mm}
\begin{flushright}
	\textcolor{PinYinColor}{\EN \huge{Fade\\
	to Black\\
	\ \\}}
\end{flushright}

\begin{figure}[!htbp]
	\centering
	\includegraphics[width=6cm]{./Chapters/Images/04.jpg}
	\caption*{清津的工业区}
\end{figure}

\ifnum\theparacolNo=2
	\begin{multicols}{\theparacolNo}
\fi
随着1990年的到来,柏林墙轰然倒地,旋即被敲成小块,变成即将统一的德国街头小贩手中热卖的纪念品。苏联帝国也分崩离析了。毛泽东的脸也出现在美国游客在北京买的那种哗众取宠、质量低劣的手表表盘上。前罗马尼亚共产党独裁者尼古拉斯·齐奥塞斯库(Nicolae Ceausescu)──金日成必然的好友,被一队士兵处决。列宁(Vladimir Lenin)雕像被从基座上推到、砸的粉碎。全世界共产党的干部们,午餐就着可口可乐、大嚼着麦当劳的巨无霸。而此时,在恍若世外的北朝鲜,一起都按照原来的轨迹继续着。\\

在某种程度,北朝鲜当局允许报导共产主义的崩溃,当然他们会就事实打些折扣,有时候还会颠倒黑白。就《劳动新闻》而言,在共产世界中其它国家所发生的问题,都归因于人固有的弱点\footnote{北朝鲜媒体总是喜欢标榜北朝鲜人生而由来的优越基因。}。东欧人、中国人意志软弱,纪律涣散。他们都变修了,偏离了真正的社会主义道路。如果在金日成这样的天才指导下,他们的共产主义也一定会焕发勃勃生机的。坚持领袖所教导的自力更生,因而北朝鲜对其他国家发生的事情无动于衷,并继续沿着自己的道路走下去。\\

所以宋女士紧闭双眼,以期对那些越来越明显的恶化迹象选择性失明。一开始,苗头很小,几乎注意不到。电灯泡熄灭那么几秒钟、然后是几分钟、几小时、最后是整夜整夜的。自来水也停了。宋女士很快想出办法,当来水的时候,家里所有的盆盆桶桶、坛坛罐罐全部用来存水。但是即使这样,还是不够用,因为整个建筑的水泵靠电驱动,每次来电的时候,水早就漏完了。因此,宋女士带上家里所有的塑料水壶,去楼下公共供水处。取水成了她每天的例行公事。每天早上,把地上的床垫卷起来,给金日成画像掸灰,之后就是取水这累人的活了。因此,即使现在孩子们都大了,她却比以前起的更早了。她上班经常搭乘的一路电车,班次现在也越来越少了,偶尔来了一辆,也是如此的拥挤,以至于车尾的梯子上都挂着人。宋女士不想在一大群小伙子中间被推来搡去的,所以通常她步行上班。她每天要花一小时才能走到单位。\\

清津的工厂沿着海岸线依次排开,从北部的浦项向南延绵13公里到罗南,那里原来是日本人的军事基地,现在成为北朝鲜人民军第六师师部所在地。其中最大的工厂包括清津钢铁厂和金策钢铁厂、化纤厂、第二钢构厂、五月十日煤炭机械厂和一个麻田鹿场,生产一种用鹿茸制成的药材。宋女士在位于工业区北部的朝鲜制衣厂清津分厂工作。清津分厂有大约2000名员工,除了高层管理人员和卡车司机,其余的几乎全部都是女工。北朝鲜人一生大部分时间都穿着制服,而这些衣服大部分都是由朝鲜制衣粗制滥造的。统一的学生装、售货员服装、列车员制服、劳工装、当然还有工厂工人的工装。所有的衣服都是有一种叫维尼龙的化纤面料制成,这种质地较硬,外表发亮的合成面料是北朝鲜所独有的。北朝鲜人对这种材料也颇引以为豪,是由一个朝鲜发明家于1939年发明的,因而也被他们称呼为“主体”纤维。这些面料大多在沿海岸线往南280公里的咸兴生产的。\\

大概自1988年开始,面料的供应开始延误。宋女士和其它工人被告知,问题出在咸兴的工厂。不是他们没有足够的生产维尼龙所需的无烟煤,就是电力短缺,没人闹得清楚到底怎么回事。但是巧妇难为无米之炊,没有面料,你就做不了衣服。\\

在等下一批原料的到货的时候,缝纫女工们就只好拖洗地板,擦拭机器。厂子里现在死一般的寂静,往日缝纫机欢快的卡搭声再也听不到了,现在能听到的只是扫地时发出的沙沙声。\\

为了让女工有些事情做,厂子的管理层开展一项工作,美其名曰“特别项目”。实际上,就是搜集一切可以卖掉或者换食物的废品。某天女工们每人一个袋子,被安排去铁路沿线收集狗粪以作肥料。另外一天,可能是拾废铁。开头,只是缝纫女工被安排出去,后来宋女士和其它日间看护中心的妇女也被安排加入其中。他们采取轮班制,中心一半的女工留下照顾孩子们,而另外一半则外出四处拾荒。\\

“即使遇到艰难险阻,我们也要誓死保卫党。”当在外面的时候,他们还这样唱到,管理者希望以此鼓舞士气。\\

有时候,他们会去海边,在那些巨型钢铁厂背面的排污管所排出的废料里,收集金属。宋女士不喜欢弄湿她的脚,即便当年和孩子们一起在靠近清津青年公园的海边拾贝壳的时候也是。像大多数她那一代北朝鲜人一样,宋女士不会游泳。即便是很浅的水,也让她胆战心惊。而现在,她却不得不卷起裤脚,只穿着帆布鞋,跋涉在海水里,用篮子像淘金一样淘着金属块。一天下来,检查人员要给捡拾到的金属称重,确保每个单位都完成了份额。\\

所有的妇女都在想方设法逃避这恼人的苦差。他们又不敢辞职,即使连工资也拿不到。在北朝鲜,如果旷工,那你就领不到粮票。如果无故1周不来上班,就会被送去拘留所。\\

有些妇女就胡编乱造些家里的急事。另外一些则弄到医生开的假条,说身体不舒服,不能来上班。其实这些大家都心照不宣,检查人员也不会仔细核对这些假条,他们也知道即使女工来上班也无事可做。然而,宋女士却不会耍这些小花招,弄个假的假条。她觉得心里过意不去。她还如往常一样,准时上班。由于缝纫女工都不来上班,因此也没人送孩子来日间看护中心。领导也就安排些关于金日成的讲座来打发时间。经过多年的一天14小时的工作,宋女士终于有机会休息一下。她可以趴在自己的桌子上,好好的打个盹,脸靠着木板,思量着这样的日子还要过多久。\\

一天,厂长把宋女士和她同事们召集到一起谈话。厂长是宋女士很敬重的一个人,他是党员,一个坚定的共产主义者,同宋女士一样,是一个真正的信仰者。在过去,他总是信誓旦旦的向工人们保证,从咸兴来的原料马上就要到货了。现在,他不自然的清了清嗓子,一脸尴尬的说道,形势短时间内不可能有什么改观。此时,她们这些和宋女士一样坚守岗位的女工们,从今以后,也不用再坚持了。\\

“阿妈们。”他说到,用了一个朝鲜语“阿妈”,这个词通常用来称呼已婚妇女的词的,“你们应该想想其它的办法给家里找吃的了。”\\

听到这里,宋女士目瞪口呆。厂长什么意思?虽然他并没有直说是卖淫,但他的潜台词很可能包含这个意思。他建议宋女士去黑市工作。\\

和其它的社会主义国家一样,北朝鲜有着黑市。虽然法律上来说,私人买卖商品是属非法,但是由于政策总是朝令夕改,因此没人把它当回事。金日成时代是允许人们在自家的自留地里种些蔬菜和出售自家的出产,因此人们在宋女士住的小区后面的空地上设立临时市场。说是临时市场,其实也仅仅是在泥泞的地面上,铺上一层塑料布的小摊位的聚集点,卖些萝卜、白菜之类的蔬菜。偶尔,也有些人卖旧衣服、残缺的陶器、二手书。总而言之,任何全新的东西是不能在市场上销售的。这些东西只能在国营的商店里出售。销售谷物也是严格禁止的,任何人如果销售大米将被判处徒刑。\\

宋女士总是认为黑市充斥着廉价、市侩的气氛。小贩们绝大多数是老年妇女。宋女士总是看见她们盘着腿坐在摊位上,面前是脏兮兮的蔬菜,粗鄙的喊着售价。有些妇女甚至不顾北朝鲜女性不许吸烟的禁忌,叼着烟杆,吞云吐雾。宋女士很反感这些老阿婆,进而恨乌及乌的对黑市买卖也是感到讨厌。那里没有共产主义。\\

实际上,纯正的共产主义是不需要商业的,即使不是绝对。金日成曾经倡导一种反消费主义的生活方式,很难想象在20世纪这样的方式居然还能得以存在。而在亚洲的其它地方,市场上充满人气,商品琳琅满目。但是北朝鲜却不是。这个国家最著名的商场是平壤的两个百货商店,第一百货商店和第二百货商店。店如其名,他们的货品也是令人印象深刻。我2005年访问平壤时,参观了这两个商场,一楼摆满产自中国的自行车,我不清楚这些商品是真的拿来销售的还是仅仅作为摆设给外国人看的。具曾经于90年代访问平壤的人叙述,这些商场有时候用塑料的蔬果蔬菜,以假乱真糊弄外国参观者。\\

北朝鲜人也不被认为需要逛商场,理论上讲,他们所需的任何物品都由政府以金日成的名义发放。一般来说,他们每年可以领到两套衣服,一套夏装、一套冬装。新衣服通常由所在的单位或者学校于金日成的诞辰日发放,以此强化金日成乃幸福之源的形象。所有分发的物品都是统一标准制作。鞋只有人造革的和帆布的鞋供应,皮鞋是非常昂贵的,只有那些有其它收入来源的人才买得起。衣服都是由类似于宋女士的制衣工厂提供。一般都是化纤面料的,这种面料很难染色,所以只有那么单调的几个颜色:工厂的工装是单调的靛蓝色,办公室人员就是黑或者灰色的。红色的面料是专门用于制作红领巾的,在北朝鲜,红领巾是孩子们围在脖子,作为少先队成员的标志,而少先队则是每个13岁以下的孩子必须参加的。\\

不仅仅是不用买东西,甚至连钱也用不着。在北朝鲜,工资更像是一种津贴,数量少的可怜。宋女士每月工资为64朝元,即使按照官方的汇率也仅值28美元,而实际上连一件尼龙织的线衫都买不到。这些钱只能当成零花钱,看看电影、理个头发、坐坐公交、买张报纸什么的。对于男人来说,也就够买买香烟。女人,也就买点化妆品──在北朝鲜有个奇怪的现象,妇女们都喜欢浓妆艳抹。大红的口红,让北朝鲜女性看起来像是40年代电影里明星的装扮,粉红的腮红掩饰着由于漫长的冬季而变得枯黄的脸。清津每一个居民区都有自己的一些国营商店,而且这些商店都千篇一律相互没什么区别。每一个北朝鲜妇女都很注重仪表:宋女士宁可不吃早饭,也不会省下化妆的时间。她的头发有点自然卷,而其它她这个年纪的女性,都去那些像生产装配线一样的理发店里烫发,理发店里一边是一排男顾客用的理发椅,另外一边是女顾客的。理发师都是国家职工,为一个叫做便民所的政府单位工作,这个所也修鞋和修自行车。\\

小区商店,除了理发店,还有食品店、服装店。不像苏联,在北朝鲜,你很少看到排队的景象。如果你想买个大件,例如买个手表、买个录音机什么的,你要先向单位申请许可。所以不仅仅是钱的问题。\\

北朝鲜体系的最高成就就是食物补贴制度。正如赫伯特·胡佛(Herbert Clark Hoover)在竞选中承诺家家锅里有鸡一样,金日成也许诺北朝鲜人一天三顿白米饭。米、特别是白米,在北朝鲜弥足珍贵。所以这也就是个空头支票,不可能对所有人兑现,只能先满足精英阶层。然而,公共分配系统确实向普通民众供应各种混合谷物,根据级别和工作内容所消耗的热量,供给的数量是经过仔细计算过了的。煤矿工人属重体力劳动,每日供应谷物900克,宋女士这样的工厂工人,每人每天供应700克。这个系统还供应其它朝鲜的日常食品,如酱油、炒菜油、一种叫辣椒酱的粘稠的红豆膏\footnote{这里应是作者有误。──译者},在国家假日的时候,如金氏家庭的生日,可能还能分到猪肉和咸鱼。\\

食物配给里,最受欢迎的是白菜,一般在秋天发放,用于制作朝鲜传统的泡菜。这种味辣、腌制的白菜是朝鲜传统的佳肴,也是漫长冬季里朝鲜人的日常饮食中唯一的蔬菜品种。泡菜和米饭一起构成朝鲜文化不可或缺的元素。北朝鲜当局十分清楚如果没有泡菜,那么幸福对于一个朝鲜人来说就无从谈起。每个家庭中,成年人可以分到70公斤白菜、孩子可以分到50公斤,在宋女士家,在她婆婆搬来和他们一起住之后,就总共可以分得410公斤的白菜。这些白菜用盐腌制,再加上很多很多很辣的红辣椒,有时候还加豆瓣酱和虾米。宋女士还做萝卜和大头菜的泡菜。做这些泡菜,宋女士要花上一整个星期的时间,并把它们储存在大的瓦缸里。长博会帮忙讲这些瓦缸搬到地下室,那里每家每户都有自己的储存柜。其实,传统的做法是要将瓦缸搬到花园里埋起来,这样这些泡菜就会保持在冷藏的状态,而且又不会上冻。在公寓楼里,他们想了个办法,就是用泥巴糊在坛子周围。当一切大功告成的时候,剩下的就是需要一把结实的锁,把柜子牢牢的锁起来。在清津,偷窃泡菜非常普遍。即使在如北朝鲜这样一个讲究集体主义的社会主义国家,也没人愿意同陌生人分享自己的泡菜。\\

很明显,北朝鲜并不是像宣传的那样是劳动者的天堂,然而金日成的功绩却也不是无足轻重的。自1945年半岛分而治之后的头20年,北边的日子一直比资本主义的南边要好。实际上,在60年代的时候,朝鲜的学者提及“经济奇迹”的时候,他们指的是北朝鲜。仅仅是喂饱历史上长期处于饥荒状态的这片土地上的人民这一点,就是个了不起的成就,再考虑到这个半岛大部分适宜耕种的土地都在南边的这个事实,这一点就更尤为可贵。在饱经战火,几乎全部的基础设施损失殆尽,70\%的房屋毁于战火,这样一个国家,在金日成的带领下,从战争的废墟中完成基本的重建。每一个北朝鲜人,有房子住、有衣服穿。在1949年,北朝鲜成为亚洲第一个宣布消灭文盲的国家。60年代,访问北朝鲜的外国政府要员,一般需要跨越同中国的边境进入朝鲜,总是津津乐道于北朝鲜明显优越的生活条件。事实上,那时候数以千计的朝鲜族中国人,为逃避由毛泽东发动的灾难性的“大跃进运动”而引发的饥荒,纷纷前往北朝鲜。那时候,北朝鲜家家都是大瓦房,70年代就村村通上了电。甚至是最顽固的中情局的分析员,也不得不承认,当时金日成的北朝鲜给人留意深刻印象。\\

作为一个共产党国家,北朝鲜更像是南斯拉夫而不是安哥拉,是共产世界引以为豪的例子。人们往往喜欢用北朝鲜取得的成就,特别是相对于南韩的成就,以证明社会主义的优越性。\\

然而真的是这样的吗?北朝鲜的所谓奇迹都是基于诸多想象,都是宣传的出来的,未经证实的。北朝鲜政府从来不公布经济统计数据,即使公布了,也没有多少可信的成分,并且政府花大力气蒙骗那些访问者,甚至是自欺欺人。因为害怕将真相报告上级,监管人员任意捏造农业产量和工业产出的统计数据的现象司空见惯。村骗乡、乡骗县、一路骗到国务院。因此可以想象,连金日成自己都闹不清楚经济什么时候会崩溃。\\

虽然强调“主体”和自力更生,北朝鲜实际上极端依赖领国的慷概援助。从邻国获得廉价的石油、大米、化肥、药品、工业设备、卡车及轿车。X光机和育婴箱来自捷克斯洛伐克、建筑设备来自东德。金日成也娴熟的利用中苏两国的对立,从双方获得额外的好处。像旧式的帝王,从邻国得到进贡:斯大林送来了高级装甲防弹轿车、毛则送来铁路车厢。\\

80年代,金日成逐步把权利移交给儿子金正日,两人都喜欢采用“临场指导”的方式处理国家问题。父子两人是从地理到农业的全能专家。“在金正日的现场指导,和亲切的关怀下,山羊的繁殖率和奶产品产量有了极大的提高。”这是朝鲜中央新闻社在金正日参观了位于清津附近的一个养羊场后所做的报导。某天,金正日突发奇想,下令将土豆取代大米作为国家的主粮;隔日,他又下令养殖鸵鸟以解决食物短缺的问题。这个国家就这样,摇摆于一个又一个的不切实际的计划之中。\\

同时,又有大量的国家财富被军队挥霍。北朝鲜的国防预算接近占国内生产总值的1/4。而相比下一般工业化国家这个比例仅仅是5\%。虽然自1953年停战以来,朝鲜半岛并无战端,北朝鲜却豢养了近100万的军队,使得这个再大也大不过这个半岛的弹丸小国拥有世界第四大军事力量。北朝鲜的宣传机器也开足马力不知疲倦的宣传着帝国主义忘我之心不死。\\

金正日在被确定为接班人之后在政治局内的地位急升,并于1991年被任命为朝鲜武装力量最高指挥官。几年之后,全国范围内的主体纪念碑傍的标语牌上开始推广着新的标语,“先军”或者“军事优先”并且宣传朝鲜人民军是一切政治决策的中心。此时小金已经羽翼丰满了,不满足于仅仅拍拍电影了,现在他的兴趣转到一个更大的玩具──核武器和长程导弹。\\

早至二战末期美国在长崎投下原子弹的时候,金日成就梦想他的国家也拥有这样的核力量,并且于60年代开始在宁边建设苏联人设计的核设施,并着手开展核武器的研制工作。但是,直到金正日时代,核武器的研制才驶入快车道。很明显如果拥有核武器,这将有利于北朝鲜在其国际影响力日薄西山的今天为自己保留一席之地。北朝鲜不是把钱投入急需重建的老旧的工厂和基础设施,相反他们投入巨资用于研制秘密武器计划,声称拥有“核威慑”是应对美国侵略必要的手段。到1989年,北朝鲜已经在宁边建立了处理工厂,从反应堆的核燃料棒里提炼武器级的钚,且据中情局评估,在90年代早期,北朝鲜可能已经拥有可制造一到两枚原子弹的核材料。“金正日不在乎这个国家是否会破产。”金斗洪,一个曾经是平壤高阶军官的脱北者于2006年在首尔的一次采访中这样告诉我。\\

而此时的时局对北朝鲜也非常不利。金正日意识到冷战业已结束,但是他却没有意识到过去的社会主义老大哥们现在只对做生意赚钱感兴趣,谁还会去投入巨资去满足一个跟不上时代的独裁政权的核野心。经济方面,他们的死敌──南韩也于70年代中期全面超越了北朝鲜;随后的10年间,更是一骑绝尘,把北朝鲜远远的甩在了后面。社会主义大家庭早就被抛之脑后,苏联和中国都更愿意同现代和三星做生意,而对总是赊账的北朝鲜国有商社则兴趣索然。在1990年解体前1年,苏联同南韩建立了外交关系,此举严重动摇了北朝鲜的国际地位。2年后,中国也跟进同南韩建立的外交关系。\\

依赖他国的北朝鲜渐渐债台高筑,截至90年代早期,累计未偿还债务达到100亿美元,债主们慢慢失去耐心。终于,莫斯科决定北朝鲜必须以世界通行的市场价而不再是社会主义盟友的“友情”价格购买苏联的出口物资。过去,供应着北朝鲜3/4的燃油和2/3的食品的中国人,现在也要求款到发货,虽然中国人一向认为两国是“唇亡齿寒”的亲密盟友。\\

很快这个国家的经济陷入万劫不复的恶性循环。没有廉价的燃油和原材料,工厂无法运作,厂子停了就意味着没有产品可供出口,没有出口就没有硬通货,没有硬通货就更没钱买燃油,进而电力无法供应。没有电,无法启动电泵抽水,煤矿无法运转。没有煤炭就更加恶化电力短缺的状况。电力短缺也影响到农业生产。没有电,集体农庄无法正常运作。即使在过去电力充足的时候,要靠北朝鲜贫瘠的土地养活2300万人口也非易事,而提高产量的农业技术却依赖电力驱动人工灌溉系统,生产化肥和农药的工厂也因缺电,缺原材料现在也陷入停顿。北朝鲜的粮仓开始见底了,随着人们食不果腹,他们也没有力气去工作,这样产量就下降的更厉害。经济呈自由落体状态了。\\

在2009年,北朝鲜是地球上最后一个所有农作物都在集体农庄种植的地方。国家征收所有农业产出,然后再将一部分返还给农民。但是随着90年代早期,连年的歉收,农民自己也开始挨饿,于是就有人开始偷偷密藏食物。有传言,发生过因将谷物藏在屋檐上而将房顶压塌的事情。农民们也在田里没有什么劳动积极性,精力都用于伺弄房前屋后的自留地或者山坡上自己开荒出的小块土地。驱车驶过北朝鲜的乡村,你可以很明显的看到农民自留地和集体大田的区别,前者满是茁壮的蔬菜,一支支豆杆朝天而起、一个个南瓜垂地,而仅仅一臂之遥的集体大田里的玉米弱不禁风,一排排歪歪扭扭无精打采的站着,这还是由那些志愿者在爱国义务劳动的时候栽种的。\\

最艰难的还是哪些无地的城市平民。\\

自结婚后,宋女士每15天就会提着两个塑料购物袋去食品配给中心。这个配给中心离家不远,就在小区里面,挤在两栋公寓楼之间。不像是在超市,你可以想要什么拿什么;女人们在这个没有任何标记的店门口的一个可以两侧摇开的大铁门前排起长龙。每家每户都安排了固定的日子。宋女士一家是每月的3号和18号。即便如此,等上个几小时还是家常便饭。配给中心里面是个小屋,没有供暖,四周刷着白灰,一个妇女哭丧着个脸,坐在铺满账本的桌子后面。宋女士递上她的粮食配额册的,一些钱和服装厂开的证明她完成工作任务的小票。营业员会计算她家的粮食定量──她和长博每天700克,她婆婆每天400克\footnote{退休人员的粮食配额会减少。},家里的孩子每人每天500克。如果家里有人出门在外,那他相应那几天的配额就会被扣减。一旦计算完成,营业员就会拿起正式印章,在红色印泥里沾沾,重重的敲在一式三份的收据上,然后交回一联给宋女士。在后面的储放着大米、玉米、大麦和面粉的仓库里,另外一个店员会称足口粮份额,然后放入宋女士的塑料袋中。\\

袋子里的东西总是让人感到意外,有时候多些,有时候少些。多年以后回顾那些日子时,宋女士已经记不清楚什么时候,她的口粮配额开始慢慢消失的,1989,1990还是1991年。当他们把袋子还给她的时候,她都不需要看就已失望至极。袋子比以前轻多了。短缺是全面性的。一个月她可能只能拿到25天的配额,另外一个月则仅仅是10天的。金日成的许诺成了一句空话,对于北朝鲜人,白米仍然是遥不可及的奢侈品。即使现在,大多数的人仍然吃玉米和大麦饭。食用油过去还零零星星的有一些,现在则彻底从袋子里绝迹了。宋女士不是那种爱抱怨的人,至少不是那种只要她想,她就能抱怨的。\\

“如果我啰嗦几句,他们可能就会把我抓走。”她后来这么说。\\

北朝鲜当局就粮食短缺提供诸多解释,有的解释荒唐可笑,有的勉强说得通。人们被告知政府现在正在囤积粮食,用于南北统一时,赈济那些饥肠辘辘的南方民众。他们还被告知,是美国长期针对北朝鲜实施封锁,使得他们买不到食物。这不是事实,但是这个说法却颇具真实性。因为早在1993年,北朝鲜威胁退出核不扩散条约时,克林顿(Bill Clinton)政府就拟实施制裁。因此金日成很容易混淆视听。他可以把污水都泼到美国──这个北朝鲜最好的替罪羊头上。“朝鲜人民长期以来一直忍受着美帝国主义的封锁禁运。”引自朝鲜《劳动新闻》。\\

朝鲜人自认为是坚韧不拔的民族──他们也确实是。宣传机器开展了一项新运动,通过回忆子虚乌有的1938年至1939年间,金日成领导一小队抗日游击战士同数以千计的敌人做斗争,在-20度的严寒下,顶风冒雪,忍饥挨饿,然而红旗却始终飘扬的经历,以期重新激起人们的自豪感。艰难的行军,他们是这么称呼这段经历的,后来被用于隐喻饥荒。《劳动新闻》号召北朝鲜人追忆金日成的奉献,希望通过此举,使人们更坚强的面对饥饿。\\

世界上没有任何力量能够阻止朝鲜人民以“艰难行军”的大无畏的革命精神,向着胜利前进,朝鲜民主主义人民共和国仍将是一个强盛国家。\\

忍饥挨饿成为人们爱国义务的一部分。平壤的标语牌也贴出新标语,“让我们每天只吃两顿。”北朝鲜电视台也播放了一部纪录片,一个男人据说因为吃太多米饭,把胃撑爆了。报纸也援引农业部官员的讲话,说按最坏情况预计,当前的短缺也是暂时性的,下一季的稻米会有大丰收。\\

当外国媒体于1993年报导北朝鲜食物短缺是,北朝鲜新闻媒体却义愤填膺。\\

国家以极其低廉的价格向民众提供粮食,以至于普通百姓甚至不知道大米的真实价格。这就是朝鲜半岛北边的现实情况。在我们的土地上,人们幸福的生活,根本不需要担心食物的问题。\\

如果北朝鲜人静下心来好好审视下那些显而易见的前后矛盾,思讨下他们被灌输的谎言,他们就会意识到他们现在置身于一个怎样的险地。他们毫无选择。他们不能离开这个国家,不能罢免领导人,不能公开表达意见或者示威。为了生存,普通人只能强迫自己不去想那么多。然后,人生存的本能让你乐观面对这一切。正如30年代德国的犹太人告诉自己在没有比这更艰难的一样,北朝鲜人告诉自己,至少我们自己独立自主。他们都认为食物短缺是暂时性的。形势会好起来的。饥肠辘辘不相信谎言,但是有时候它也会被骗。\\

随着新的宣扬运动的展开,当局加强国内监视网络。怨言越多的地方,那么确保无人胆敢抱怨也就显得愈加重要。\\

自70年代早期,宋女士会定期时不时的担任人民班长──她所在邻里组织的负责人。每年街坊邻居们都会选出一个负责人,通常由已婚的中年女性担任。宋女士是这一职位的不二人选,她精力充沛、组织纪律性强、对党忠诚、还具备朝鲜语称为Nunji的一种素质,大致可以译为直觉。她同每个人相处都很好。作为负责人,她要将所有小区事物列表,然后在将这些工作分配给所辖的十五个家庭,包括清扫人行道,打理公寓楼前的草地,收集可循环利用垃圾。还有一项工作就是将辖区内的可疑情况报告上级。\\

宋女士向一位国家安全保卫部的官员报告工作。江同志,是一个年长宋女士几岁的妇女,听说丈夫是一个在平壤颇有人脉的劳动党官员。每隔几个月在地区办公室,江同志就要听取宋女士就邻里情况的报告,有时候江同志也去宋女士的家,边听情况边喝家酿的米酒。但是一般宋女士没什么好报告的。公寓楼里的生活很平静。没人惹什么麻烦,除了长博抱怨雨鞋的那次。\\

后来江同志变得比以前更加坚持。当食物的配给变得越来越少的时候,她就变得更想知道人们是不是因此对政府牢骚满腹。\\

“他们对食品短缺有没有怨言?他们都说什么?”江同志问道。她在公寓楼前等了很久,最后把宋女士堵住在入口处。\\

“他们没说什么。”宋女士有点反感的回答。这是真话。实际上,宋女士已经注意到当她走进邻居的家时,人们的对话就立刻停止了,只剩下令人尴尬的沉默,而且无论她走访那个家庭都是一样。每个人都知道她是人民班长,专门负责向国家安全部打小报告的人。\\

江同志对这个回答很不满意。\\

“你应该先发些牢骚。你应该问为什么食物配给没有了。看看他们什么反映。”说这些话的时候,她压低嗓子左右看看,确保在她说这些话的时候门厅上面的楼梯里没有人听见。\\

宋女士微微点点头,此时她只想夺路而逃。她并不想用这种方法。她知道邻居们里没人涉及危害国家行为。他们不是国家的敌人。她只是仅仅太累了而不想再去考虑什么意识形态。\\

食物缺乏也慢慢使她失去往日的充沛精力。现在她时时刻刻都全神贯注与一件事情,脑子想的全都是这个,任何其它的东西都都要靠边站,这个事情就是如果给家里找到吃的。服装厂在1991年彻底停摆了,在最后一整年,宋女士都没有领到过工资,只有食品券,然而由于公共食品配给中心根本没有吃的,食品券也成为废纸一张。在过去,宋女士的丈夫时不时还能因为加班而获得些额外的食品礼物──有时候是食用油、饼干、香烟或者白酒──但是现在也没有了。国营商店的货架上也是空空如也。\\

在工厂关门之后,宋女士不得不将自己的对黑市的禁忌抛之脑后,现在只有黑市才有食物,有时候甚至还有大米,不是价格确是高的离谱。黑市上,每公斤大米大概要25朝元,而在食品配给中心,这个价格只是一毛钱而已。\\

宋女士对于要在市场上找活做觉得心里没底。她能做什么呢?她没有地,种不了蔬菜去卖。商业技能上,除了能用算盘什么都不会。要养四个孩子,况且刚刚大女儿出嫁,家里没存下几个钱。她寻思着把家里值钱东西卖掉。在她脑子里她把家里值钱的东西过了一遍,首当其冲的就是电视。然后是丈夫的那些存书,兴许缝纫机也可以换几个钱。\\

当宋女士盘算这些的时候,数以千计的其它人也在做着同样的事情。他们有什么可以卖的呢?在那能找到吃的呢?\\

清津基本上是个混凝土丛林。只要不是陡峭的山坡,早就用来盖上房子,或者铺设道路。因此也没有树林,你可以去捕鸟或者采摘野果。宋女士一家曾在去捡拾贝壳的海滩收获也很有限,沿岸的海水也很深,基本也钓不到什么鱼。城里唯一比较适宜种植的地方就是位于罗南的一个小水湾,那里有一些菜地和水稻田。\\

人们开始去更远的野外找吃的。镜城县有个果园,那是人们争相前往的地方。在周末的时候,清津的家庭们,成群结队的徒步南下,目标就是离市中心5公里之外的果园。当然还要装作是全家郊游的样子,没人愿意承认这么做是因为饥肠辘辘。果园是一个集体农庄的,里面种植着特有的朝鲜梨,专供出口日本换取硬通货。朝鲜梨的大小形状有点像葡萄柚,但是有着波士克梨的红褐色,又有着苹果一样的甜脆。这种通体浑圆的果实成熟后,常常从树上跌落下来,有的就会滚到围着果园的栅栏之外,人们很容易就能捡到。来捡果子的大多是孩子。由于学校供应的午餐越来越少,最后干脆不见踪影,孩子开始逃学,到处找吃的。他们很容易从栅栏下面的缝隙钻过去。有个年轻人,在1992年的时候才10岁,颇为得意的回忆当年他趴在公交车的后面,然后在位于终点站的罗南跳车,再步行一小时。由于年纪小、加上就一个人、没人会注意他。他瘦小的身体让他很容易就从栅栏下挤进果园,然后拿得动多少就拿多少。“我有多少就摘多少,回来后分给我的朋友们。”他说。\\

这个时期的其它记忆大多是苦涩的。金智恩,一个刚刚从医学院毕业的实习医师,于周末和她的父母、姐姐、姐夫还有两个很小的孩子一同去果园。一路上带着怨声载道,才刚刚蹒跚学步的孩子们,直到下午3点左右才走到果园。在他们之前,已经有太多人来过了,以至于他们在地上只找到一个有点烂的梨子。他们把梨带回家,煮熟之后,切成了五份分给了孩子,父母和金智恩的姐夫,智恩和姐姐都没有分到。\\

日期是1993年9月9日,金智恩永远不会忘记这个日子,因为那是她一生中第一次,一整天没吃任何东西。其它人很少能这么清楚的记得这个日子。因为一个时代的终结不是突然一下子。很多人花了1年的时间才意识到他们的世界已经不可逆转的改变了。\\
\ifnum\theparacolNo=2
	\end{multicols}
\fi