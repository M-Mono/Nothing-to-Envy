\fancyhead[LO]{{\scriptsize {\FA \ }我们最幸福 {\FA } 年后记}}%奇數頁眉的左邊
\fancyhead[RO]{{\tiny{\textcolor{Gray}{\FA \ }}}\thepage}
\fancyhead[LE]{{\tiny{\textcolor{Gray}{\FA \ }}}\thepage}
\fancyhead[RE]{{\scriptsize {\FA \ }我们最幸福 {\FA } 年后记}}%偶數頁眉的右邊
\fancyfoot[LE,RO]{}
\fancyfoot[LO,CE]{}
\fancyfoot[CO,RE]{}
\chapter*{2015年后记}
\addcontentsline{toc}{chapter}{\hspace{5mm}2015年后记}
\vspace{10mm}
\begin{flushright}
	\textcolor{PinYinColor}{\EN \huge{2015\\
	Epilogue\\
	\ \\}}
\end{flushright}
2011年12月19日正午,北朝鲜的广播和电视发布了一则特别的消息,宣告金正日已死于心脏衰竭。他在相对年轻的69岁与世长辞,而他的死并不完全在意料之外。两年半前,他就中风了,走起路来步履蹒跚,一双手臂明显麻痹,一度饱满的大肚腩也消了气,一连数月没有出现在公众面前。平壤按照金日成于1994年辞世后的处理方式处理每一个细节。在他死后,北朝鲜政府给自己两天的时间做淮备,接着通报所有相关单位、军队、学校与官方机构即将发布特别公告的消息。电视主播李春姬穿着似乎是之前同一套的黑色传统服装,以颤抖的哭腔播报金正日之死。为期10天的国丧期正式展开。也和之前一样,平壤的电视连续播出哀戚的群众在全市各铜像聚集的镜头,只不过这一次是穿着冬天的大衣。街上有着希希嗦嗦的低喃声,伴随着抽泣和啜泣,间或传来一声声的“Abogi、Abogi”,或者“父亲”。葬礼的仪式包括在平壤街上长达3小时的游行。纷飞的白雪覆盖街道,某位播报员说是“从天堂落下的眼泪”。带领送葬队伍的是一辆黑色大礼车,车上架着一副面带微笑的金正日肖像,肖像有广告看板那么大。后方,另一辆大礼车载着棺木,政府高官在两旁随行。前方,身着一袭黑衣、一手扶着礼车的,是个胖都都的青年,这人还不满30岁。\\

金正恩刚成为全世界最年轻的国家元首。他从不明朗的处境中一跃而出,把金家王朝延续下去。身为身分受到承认的金正日的第三子,他是晋升领导者的一匹黑马。小时候,他假冒成北朝鲜大使馆某位普通外交官的儿子,被送到瑞士的伯尔尼(Bern)念中学。金正恩本来不是什么重要人物。金日成假定的继承人是长子金正男,但他因为2001年持假护照赴日本迪士尼乐园游玩被捕,令北朝鲜蒙羞而丧失了资格。于是,排名最后的金正恩被带回北朝鲜,伴着他的是一份捏造得天衣无缝的学经历,以及新的一页神话。据称他有金日成综合大学(Kim Il Sung University)物理系的学位,还有金日成军事学校(Kim Il Sung military academy)的学位。政令宣导人员在2009年开始推出这位新的偶像。首先,他们在意识形态训练课程中以“年轻将领”和“杰出同志”介绍他,直到第二年年他成为四星上将及中央军事委员会副委员长才提及他的大名。这就等同于出道发表会。2010年10月10日,金正恩站在父亲身旁,出席劳动党65周年庆的盛大阅兵典礼,在公众面前亮相。\\

金正恩是个身材浑圆的年轻人,以年仅30岁的人来说,他的腰围和双下巴颇为突出。他那像猫王般顶部往上梳、两侧剃光光的发型,就如同他父亲的增高鞋和墨镜一般,惹得讽刺漫画不画他都不行。年轻的金正恩的最佳招牌是笑容,露出一口白牙,还有个小酒窝,让他显得和祖父有几分神似,而又惹人喜爱。在外人眼中,这位少年头家的体重在一个濒临饥荒的国家显得很突兀,但北朝鲜人似乎只因此而更尊敬他。\\

金正日死后的馀波,是我第一次听到北朝鲜人民表达出一丝乐观的迹象。不需要太多的刺探,北朝鲜人就会承认他们对金正日的观感是五味杂陈;他们把饥荒归咎于他。“金日成死时,我哭得死去活来。我不知道我们要怎么活下去。金正日死时,我也哭了,但没哭得那么惨。我们的生活这么艰难。老实说,我们对他没那么忠诚。”2012年,我在中国遇到一名来自平壤的女性,她告诉我:“他这么年轻。我们认为他会开放北朝鲜。他不会像其他人那样治理国家。”\\

金正恩此时上任可谓相当走运,一大堆和2012年金日成百岁诞辰绑在一起的建设计画已经动工。数以千计的住房单位有着现代主义大师柯比意所设计的流线形外观。大学生被拖出学校来当盖房子的“自愿者”。柳京饭店这栋105层楼高的金字塔型建筑,还没盖完就空荡荡地闲置了超过20年,俨然已成为全国的笑话,如今也再度开工。平壤火车站上方,一块时代广场风格的电子看板播送着北朝鲜的电视节目。在这个从80年代以来就没什么改变的城市,亚洲最故步自封的首都之一,这副景象相对更显壮观。建造工程实际上从2008年就展开了──那年,我很讶异在一次旅途中听到电钻的声音──但这一切却让人觉得是精力旺盛的年轻领导人的杰作。\\

成为领导人不久后,金正恩娶了年轻迷人的李雪主为妻。根据某些报导,她来自清津。她和她的领导人丈夫常被拍到一起公开亮相──她往往身穿香奈儿风格的订制套装,而最引人注目的是,她没有配戴金日成肖像的徽章。在过去,领导人的配偶是国家机密。她的公开形象犹如一大进步,让这个国家显得稍微不那么奇怪。\\

至少在一开始,金正恩表现得比他父亲对经济改革的态度更为开放。金正日最后几年执政的特色是几乎持续不断地打击市场──政府频频禁止大豆、马铃薯、化妆品或任何“中国制造”的东西。2009年底,更以迅雷不及掩耳的速度作废北朝鲜货币,借以控制市场。此举引发连月溷乱,几乎掀起暴动。金正日死后,压力减轻,市场获淮更为自由地运作,民众稍微松绑。北朝鲜官员会私下告诉高层代表,新的领导人会将经济摆在第一顺位。2012年夏天,一名资深人道救援官被告知:“党与人民团结的力量远远强过原子弹”、“为了发展敝国经济,敝国需要和平的气氛。”。\\

情况还是一样,金正恩并没有要背离父亲的武器计画。2012年4月,北朝鲜尝试发射人造卫星,基本上是和洲际弹道导弹一样的技术,它在升空后几秒钟就坠毁了。12月,他们又试了一次,这次成功把一颗小型卫星送上轨道。接着,2月时,他们宣布他们已经在距离清津60里的吉州完成一次地下原子弹试爆,那是2006年以来的第三次测试,多多少少证实了北朝鲜至少也拥有粗糙的核武技术。这本来可以是金正恩庆祝胜利的一刻,结果却演变成一场公共关系的灾难。面对联合国不可避免的原子弹试爆制裁行动,北朝鲜做出犹如集体精神崩溃般的反应。它撕毁1953年终结韩战的停战协议,对韩国宣战,威胁要以核武攻击美国和美国在关岛与太平洋的基地,警告说他们会“扭断丧心病狂的敌人的手腕,彻底切断他们的气管,让他们清楚看到真正的战争是什么样子”。2013年4月初,北朝鲜要求外国大使馆从平壤撤离,因为这个区域“就要掀起原子弹大战”。\\

即使以北朝鲜的夸张标淮来说,这也够吓人的了。全世界的头条新闻叫嚷着重启韩战的可能,这一次说不定还牵涉到原子弹。美国增强武力严阵以待。北京气急败坏地指责年轻傲慢的金正恩把更多美国军队引来太平洋。中国一反过去对北朝鲜的支持,在联合国安理会投票赞成制裁行动。中国学者公开表示中国应该终止对北朝鲜的支持,这可是让北朝鲜的前景蒙上阴霾,因为截至2013年为止,北朝鲜有大约九成的燃油都仰赖中国进口。雪上加霜的是,北朝鲜没来由地决定关闭非军事区北边的开城特级市工业园区。在这座工业园区,北朝鲜劳工受雇于韩国人经营的工厂,一度展现了南北朝鲜之间的“阳光政策”,也是北朝鲜其中一个最稳定的合法收入来源,每年供应9000万美元的薪资。一般认为,北朝鲜政权是想藉由经过精打细算的威胁得到注意,最终获得援助与让步。换言之,北朝鲜政权是个理性而高竿的演员。这下子,威胁攻打美国的北朝鲜,却显得像“河东鼠吼\footnote{典故出自电影河东鼠吼(The Mouse That Roared),片中的迷你小国对美国宣战,打了一场糊涂仗。}”的老鼠一样滑稽。\\

此举引发的反效果始料未及,那些威胁太言过其实,并不符合北朝鲜实际上的能力,人们开始质疑新政权走不走得下去。一位退役的中国将领告诉我:“这孩子不知道自己在做什么。”\\

金正恩的行为只变得更加阴晴不定。截至目前为止,最莫名其妙的一起事件要属在2013年12月铲除并处决他的姑丈张成泽。67岁的张成泽在咸镜北道长大,也有人说是清津。在神魂颠倒地坠入爱河之后,他违背金日成的意愿,娶了金正日的小妹、也是唯一同父同母的手足金敬姬,他们的罗曼史是清津热门的八卦话题。年轻时冲劲十足的张成泽到过韩国和中国,成为这个封闭的统治家族当中最见多识广的一位。他把他的两个哥哥安插在高阶将领的职位,又为侄子和妻舅安排外交职位。他透过军事贸易公司,一手掌握海产、煤炭、矿物和民生消费品等边境贸易,被认为是推动国境之北商业活动的功臣。在他的最后几年,意识到自己健康状况衰退之下,金正日提拔张成泽为实质上的摄政王,辅佐年轻的继承人步上轨道。在这之后,他被认为是举国上下权位第二高的人。\\

此番肃清行动之戏剧化足以让史达林引以为豪。北朝鲜电视播出劳动党一次特别会议的画面,会议中张成泽当场被拖了出去。几天后,新闻报导说“连狗都不如的卑鄙人渣张成泽”已因意图掌权遭到处决\footnote{中国某些博客声称张成泽被脱个精光,活活喂给一群饿狗。不过这个说法几乎可以确定是空穴来风,比较有可能的情节是张成泽单纯只是遭到枪毙。}。一篇非比寻常、长达2700字的报导指控张成泽将天然资源贱价卖给中国人,并对金正恩表现出种种不敬的行为。举例来说,张成泽被控在金正恩升任中央军事委员会副委员长时“拍手拍得意兴阑珊”,并且在内卫部队办公室将金正恩的铜像放在阴暗的角落,而不是放在见得着光的地方。后续报导指出张成泽垮台的真正原因是他独吞中国边境贸易的大饼,与重要的军事伙伴做出切割。\\

尽管没那么大张旗鼓,金正恩后续还处理掉多位前朝元老。截至2013年底,当初为金正日扶灵的七位大臣,已有5人遭到肃清。似乎因肃清张成泽而获益的副委员长崔龙海,也在2014年5月1日遭到降职。金正恩仿佛是有计画地从政权中踢掉这些人,同时拉拔他自己的手足──主要以身为艾瑞克克莱普顿(Eric Clapton)的歌迷、并追随他在世界各地的演唱会而出名的哥哥金正哲,以及经常现身陪同其侧的小妹金汝贞。\\

在他的统治之下,金正恩以追求年轻化为重。身为新任领导人,他的其中一个当务之急就是监督平壤的老旧游乐园翻新。北朝鲜的政令宣导人员广为散布他坐在新云霄飞车上的照片。在阅兵表演和施放气球的助阵之下,一座新的水上乐园于去年开张,里面有红、黄、蓝相间的滑水道;全国第一座滑雪场于一月开始营运。至于与国际上的接触,在金正恩统治下获邀到平壤的外国人,最受瞩目的要属浑身刺青、脸上穿环的篮球员丹尼斯罗德曼(Dennis Rodman)。他在2014年1月8日醉醺醺地出席金正恩的31岁庆生会,高唱生日快乐歌,令人毛骨悚然地联想到玛丽莲梦露(Marilyn Monroe)给约翰肯尼迪(John F. Kennedy)的生日祝福。\\

北朝鲜一如既往地和国际社会脱节。联合国人权委员会2014年2月发布的北朝鲜人权调查报告,对这个国家做出了截至目前为止最为全面的控诉。在这份长达400页的报告中,委员会指控北朝鲜“屠杀、谋杀、奴役、刑求、监禁、强暴、强迫堕胎及施行其它暴力,以政治、宗教、种族、性别为由进行迫害、强迫人口迁徒、强迫人口失踪以及蓄意导致长期饥荒等不人道行为。联合国专桉小组形容这些惨无人道的罪行“在当今世上无出其右者”,并提高领导阶层应受国际刑事法庭审讯的可能性,甚至包括金正恩本人在内。\\

再来,北朝鲜也牵连上近年来最具毁灭性的网络攻击。骇客入侵索尼影视娱乐(Sony Pictures)的电脑网络,窃取珍贵的机密资料,包括令人难为情的私人电子邮件。这起事件成了接下来好几个月的头条新闻。北朝鲜此举显然是为了报复塞斯罗根(Seth Rogen)的讽刺电影《名嘴出任务》(The Interview),片中一名电视记者被派去暗杀金正恩。就某方面而言,这次的网络攻击是金正恩截至目前为止做得最成功的一件事;在2015年第一季,索尼必须拨出1500万美元的预算做危机处理,而这次的攻击比起任何核武或导弹测试都让金正恩获得更持久的关注。\\

有关解除核武的谈判始终陷入僵局,多位谈判者都做出金正恩永远不会愿意放弃发展核武的结论。2013年,他在一次劳动党中央委员会上公布他的口号──“Byungjin”,大概可以粗略翻译成“同步政策”,宣告北朝鲜将同步发展经济与核武实力。为了推广这项新政策,他们制作了一首歌和一部欢乐的政令宣导影片,导弹和滑水道、坦克与工厂的画面在片中交替出现,搭配好记又动听的副歌:“经济和核武同步前进。”\\

美国的分析家对这种想法嗤之以鼻。“他妄想鱼与熊掌兼得。”一位美国官员对我说。但北朝鲜却微微显露出经济复苏的迹象。在2013年和2012年,北朝鲜经济实际上呈现略微成长的态势。根据韩国中央银行发布的数据,北朝鲜这两年的经济成长率分别是1.1\%和1.3\%。位于首尔的现代研究院(Hyundai Research Institute)预测2015年将提升至7\%。至少直到张成泽的肃清之前,北朝鲜都还在研拟13个经济特区,这些计画就彷效中国在80年代的自由贸易实验。\\

北朝鲜在过去10年所发生的头号大事或许是手机的引进。2008年,负责翻新柳京饭店的埃及电信商奥斯康引进了这项服务,据报在2013年北朝鲜国内已有200万部手机。要装电话也变得容易得多。对北朝鲜来说,电话是一种启蒙,它就算没将这个国家带到21世纪,至少也带到了20世纪中期。尽管电话不能用来打到国外或搜寻网络,它至少也把这个国家往前推进了几十年,让它得以正常运作。市场里的商贩如果需要更多存货,或想要知道在国内其他地方的商品价格,只要打一通电话给供应商就可以了。在此之前,就连想要完成最简单的任务都会困难得令人抓狂。\\

金正恩试图要玩与中国共产党一样的把戏──试探一下开放经济的效果,但依旧紧握政权不放。尽管在经济上有种种补破网的作为,但思想的自由和表达还是付之阙如。\\

北朝鲜依旧是2300万人民的牢笼,只不过笼子里的条件可能有所改善,至少对平壤的核心阶层而言是如此。金正日和金正恩显然明白他们的存续有赖于死忠拥护者的效忠。平壤的高层干部过去顶多拥有和韩国工厂工人一样的生活水淮,如今菁英分子至少也有舒适的公寓。如果他们有钱,也有民生消费品可买。高级商场曾经被视为耻辱,但现在北朝鲜的电视会播出豪华的开幕典礼。就连金正日都拥抱消费主义:在他死前最后一次的公开露面,他去平壤逛一家由中国人经营的沃尔玛,那间超市有20种牙刷、12种不同牌子的啤酒,以及像是四季宝花生酱(Skippy Peanut Butter)这种进口货。近来参观平壤的外国人会很讶异地看到,这座首都并不符合它斯大林主义时间胶囊的形象。这座城市的年轻女性穿高跟鞋,青少年把棒球帽反过来戴,小女孩穿着漂漂亮亮的粉红色。平壤有迪斯尼卡通人物T恤,也有愤怒小鸟背包。而北朝鲜最新的流行是直排轮──金正恩的年轻化政策所引进的另一项产物──一样也为这个国家营造出一个幸福生活的假象。\\

清津最近发生了高楼窜起的现象。几年前,一号道路上的建筑外墙做了平整,有些建筑也展开整修。浦项广场的金日成铜像旁,去年开工建造一栋22层楼高的大楼。截至去年夏天为止,他们已经盖到12楼,尽管工程似乎停摆了。广场后方是一个展览中心,里面有模型展示一座计画中的游乐园,以及一座被一位参观者形容是“未来派杜拜风格的高塔”。\\

这看来是为了吸引观光客所费的一番工夫,而吸引观光客则是希望他们能带来强势货币。在最近的参访行程中,学者被护送去参观一所模范幼稚园,那里的女童脸颊和嘴唇涂得红通通,穿着萤光粉红色的裙子,在舞台上娱乐观众。模范儿童盛装打扮给外国人看。但当游客走出幼儿园的大门,从公车上看过去,他们瞥见两个大约和里面的幼童年纪相当的小男孩,穿着肮葬、过大的衣服,正在徒手挖着一堆石头。\\

北朝鲜观察家花很多时间彼此争论这个国家的国内情况究竟是比较好、比较差或根本没有改变。但我们的观察都存有疑点,因为北朝鲜在蒙蔽真相上下了不可思议的苦工。观光客会在各种不可能的场所,看见盛装打扮的可疑人物在那边摆姿势。年轻女性坐在金日成铜像前的水泥长椅上假装看书,她们的脸颊抹了鲜艳的腮红,身穿传统服饰,方方正正的上衣系了一个蝴蝶结,僵硬的裙摆膨得像帐篷似的。事实真相只在转瞬之间。有一次,在雕像前,我从后面看着一群军人代表,穿着光鲜体面、烫得笔挺的制服上前献花。当他们深深一鞠躬以表敬意时,他们的裤管提了上来,刚好可以让人看到他们没穿袜子。\\

2005年初次造访平壤时,我在晚上回到高丽饭店的房间,发现尽管有标语建议宾客节约用电,服务生却把每一盏电灯都打开了,包括浴室和衣柜里的电灯。后来有人向我解释,由于适逢中国国家主席胡锦涛造访,灯火通明是为了给他的代表团一个好印象。2008年,我再度造访平壤,这次是一个代表团伴随纽约爱乐交响乐团而来,只见全城张灯结彩,仿佛正值耶诞佳节。金日成广场沐浴在探照灯下,白色小灯编成的花圈让主要街道明亮起来。包括音乐家和记者在内,人数超过300人的代表团住在羊角岛国际饭店。虽然是2月,外面冷得要命,旅馆房间却热到我们脱得只剩恤。他们设了一个可以连上网络的媒体采访中心。晚餐是有鲑鱼、奶油烤螃蟹、羔羊肉、薄切稚肉和维也纳风格巧克力蛋糕等多道餐点的盛宴。我们的自助式早餐台装饰了冰雕和果盘,食物也相当丰盛──当中还包括香蕉三明治──或许有点怪,但仍旧不失为一场华丽的大秀。就连我们当中最为心存怀疑的记者都不禁认为北朝鲜要出头天了,它正稳步脱离90年代的困境。当然,我们被骗了。那只是海市蜃楼,是一片漆黑中一道稍纵即逝的光芒。北朝鲜的真面目是一个失能的国家。网络连结消失了。万家灯火熄灭了。音乐会过后1星期,我和当时在平壤的联合国世界粮食计划署代表尚皮耶德玛杰里通电话,他告诉我:“你们一离开,这里又是漆黑一片。”\\

只要离开平壤,真实的北朝鲜就映入眼帘,尽管只是透过客运巴士或快速移动的汽车车窗才能看见。在南浦特别市\footnote{美兰初次亲眼目睹尸体的那座西岸城市。},这是我第一次瞥见流浪的燕子──一名年约9岁的男孩,穿着一件葬兮兮的制服,光脚走在路边。显然无家可归的人们,就睡在马路旁的草地上。在一般工作日的早上10点,不少人弯腰驼背蹲坐在路旁,他们垂着头,一副无事可做的模样。近来我在中国遇到的北朝鲜女性大致和宋太太很像,都是疲于工作勉力维持家计的伶俐人。\\

她们有些是劳动党员,运用人脉或金钱到中国来工作。我在2012年被介绍认识金槿姬,她是年纪50开外的妇女,皮肤皱得像陈年羊皮纸,粗硬的卷发绑在脖子后面。她告诉我她是怎么协助她先生和两个成年儿子酿私酒和养猪的──这两件工作搭配得天衣无缝,因为她就用酿酒产生的酒糟喂猪。她凌晨4:30起床,拔草为她自己和猪加菜。她从不吃肉,他们会把猪卖掉去买米,把米加到做为他们主食的玉米葫里。金太太说,她25岁的大儿子在从军7年之后被军方解职,因为他营养不良病倒了。“他一餐只能吃到三个土豆。没有米。”她的小儿子20岁,被分配到一家工厂制作铁路设备,但由于没有薪水,他实际上还每个月付给工厂3块钱,好让他能开小差去帮他妈妈养猪、酿酒。\\

金太太来自平壤郊区,定期会进城去,她看得出来平壤在进步。“有更多建设,更多人在盖房子,平壤也有更多东西可买。但日复一日,我们的生活其实更困难。”金正恩似乎把他的好意都挥霍在他的浮华计画上了。“大家在挨饿的时候盖一座游乐园有什么用?”58岁的卡车司机金永哲质疑道。他也来自平壤郊区,于2013年8月脱北。\\

“金正恩上任时说他会改善人民的生活,可是这件事根本就没有发生。”近年来,我所遇见的每一个北朝鲜人都描述了暴力犯罪和吸毒犯滥的现象。在咸兴市遮得密不透风的制药厂附近,失业的制药师于2004年左右开始在厨房实验室里炼制甲基安非他命。从那之后,北朝鲜人称之为“Orum”或“冰毒”的甲基安非他命,就散布到清津和像会宁这样的边境城镇。它既便宜又能抑制食欲,对北朝鲜来讲是理想的毒品。北朝鲜人告诉我,它被任意当成用来招待客人的东西。主人不是请客人喝杯茶,而是哈一口。\\

随着贫富差距越来越大,无业游民有增无减,这当中不只有流浪儿童,也有流浪老人。“年轻人自身难保,有时只好把老人扫地出门。”一名于2012年夏天脱北的49岁妇女说。这名妇女名叫朴贞淑,她向我提到在脱北几个月前,她在水南河岸从一具无人认领的老人尸体旁边走过。“一直走到火车站,我还是看到很多无家可归的孩子。你得把你的食物遮好,否则他们会来抢。”\\

跟宋太太很像,朴太太也是自食其力。她靠烤饼乾维生,但由于买不起糖,她只好添加糖精让饼乾变甜。她先生在工厂有一份无薪的工作。朴太太告诉我,她是劳动党党员,并运用她的人脉获淮造访中国,希望能来跟亲戚借点钱帮助家里。\\

我问她,她觉得还有多少人依旧是北朝鲜政权的信徒?她压低声音,毫不含糊、斩钉截铁地说:“没有。让我们支撑下去的不是对体制的信心,而是对活着的信心。”\\

其中一个体制衰败的迹象在于:如今金钱比政治上的效忠重要得多。人们可以靠金钱越过边境或逃出劳改营。讽刺的是,脱北者的家庭通常比一般人生活优渥,因为他们有韩国送来的钱。帮人偷渡出去的掮客也赚到了现金,而这些现金则用来买公寓、电脑或只是买米。渐渐地,保安人员都知道哪些是脱北者的家庭,并且登门索贿。玉熙告诉我,街坊上的保安人员需要刮胡子的时候就去找她老公。“他们知道他是唯一一个有刮胡刀的人。”玉熙说。\\

北朝鲜政权还能维持多久?这是我在公开谈论北朝鲜时最常被问到的问题,也是我在首尔出席的无数晚宴上,记者、学者和外交人员谈话的主题。\\

尽管有种种不利条件,北朝鲜撑过了柏林围墙倒塌、苏联解体、中国的市场改革、90年代的饥荒、金日成之死、小布什(George W. Bush)总统的两任任期──小布什(George W. Bush)将北朝鲜列为“邪恶轴心国”,并威胁要让金正日遭到跟萨达姆(Saddam Hussein)一样的下场。金正日不只让他气数已尽的预言落空,还成功将权力巧妙移交给他那年轻、生涩的儿子。\\

北朝鲜持续破除有关它垮台在即的预言。许多分析家预期,在金正恩肃清姑丈张成泽之后,北朝鲜政权会因为内部斗争而导致内爆的结果。但目前看来,这个年轻人依旧稳坐王位,犹如还在冷战时期似地治理他的国家,大量滥造夸张的政令宣导,禁止多数外国人造访,以核武和导弹威胁真实的和想像的敌人。他活脱脱是当今之世硕果仅存的独裁者,也是一个活生生的时代错误。\\

北朝鲜政权的延续对脱北者而言是一大绝望。今年稍早,我最后一次看到俊相时,他告诉我他对北朝鲜垮台不抱希望,也不期望它有什么重大改革。他已经接受这辈子再也见不到父母的事实。“事情的发展令人无法置信。我不认为我们会有必要开放边境,但我不能想像在过了10年之后,我们还是不能传简讯或通电话。”俊相有点苦涩地说:“北朝鲜照理说有上百万部手机,但如果你不能打到国外,那它就是个笑话。”\\

以本书当中和家人有所联络的脱北者来说,他们是透过在茂山和会宁等城镇使用非法中国手机拨打,这些地方靠中国够近,能够接收到边境那头的讯号。为了打手机,他们的家人必须从清津过来,旅途所费不赀,而且越来越冒险。自从张成泽去年遭到肃清以来,北朝鲜政府显露出制裁跨境与中国交易的迹象。家人害怕被监听,不敢多说,对话通常很简短而无法令人满足。\\

“他们总是说情况不好,寄钱过来。”宋太太说。她每年和住在清津的哥哥姐姐联络几次。\\

我和这本书里的六个人仍保持联络。我认为他们过得都比一般脱北者要好,而这或许并非巧合。至少在潜意识里,我不自觉地挑选了比较乐观的样本来勾勒脱北者的面貌;我也认为他们之所以愿意对我敞开心房,意味着他们已经能够消化脱北的冲突情绪,又或者聊一聊具有能让他们宣泄一下、走出阴霾、继续前进的作用。\\

从妹妹、妹妹的丈夫到孩子,玉熙一个接着一个将家中的晚辈带了出来,每次付给人口贩子1万美元。玉熙经过一番苦劝、花了一笔钱,才把自己的女儿弄出来。现在,玉熙经营3家Karaoke。每个人都在家族事业里工作,包括宋太太在内。尽管年近70,宋太太已经可以退休了,但她说:“我怕不工作会太无聊。”\\

经过几年的苦熬,金智恩通过医疗执照考试,在韩国成为一位合格的医师。她在东首尔开了一家诊所,收入足以让她付钱给掮客,把她的儿子从北朝鲜弄出来\footnote{她的前夫已死于癌症,所以没人阻止他离开。}。她夸口说她儿子“聪明、高大,在和一位韩国女性交往”。上次和她谈话时,这个年轻人和他的教会唱诗班去了夏威夷。金医师自己也四处旅行,善加把握脱北者所能享有的机会。去年她去了柏林,对德国分裂又统一的故事很是着迷。“我很讶异有些东德人说以前的日子还比较容易,因为你会按照技能分派到工作,他们有些人费了一番工夫重新适应。北朝鲜和韩国也会像那个样子。”除却她的成功,她说她还是觉得自己很北朝鲜。“听到韩国人骂北朝鲜,或者说领导人坏话的时候,我内心深处还是一阵刺痛。那甚至是无意识的,就是我这个人的一部分。”我是在2月16日和她说上话的,我问她知不知道那天是什么日子。她在电话那头犹豫了一下,接着哈哈大笑起来。“喔,我的老天,我的老天!”她用英文重复说着。那天是金正日的生日,北朝鲜日历上一年当中最重要的一天。“我不敢相信我忘得一乾二净。”\\

美兰去年搬到江南,这地方是首尔一个繁荣热闹的新兴区域,因为朴载相的《江南Style》而声名大噪。她生了另一个宝宝,这次是个女儿,再加上本来的儿子,组成一个圆满的家庭。她和她的孩子常常拜访她先生在加拿大的亲戚,好让孩子可以学英文。今年稍早,我们共进晚餐时,我很兴奋我们至少可以用一点我的语言沟通。比起这本书里的其他人,美兰最没有脱离她对北朝鲜的自我认同。每个月有几次,她会开车到首尔北边的一间新生活适应中心,指导初来乍到的北朝鲜人怎么找工作、买公寓、融入社会。“我猜我是其中一个成功的故事。他们听我谈话,而我的话让他们在适应过程中受到鼓励。”美兰说。\\

俊相在首尔过着平凡而平静的生活。他回避辅导北朝鲜人的教会和互助团体。他不公开谈论北朝鲜,以免殃及他的家人。他经营一门小生意。继一开始对穿牛仔裤和留长发的热情之后,现在他爱的是专业男性意气风发的形象──麻料西装外套、欧洲风格的墨镜、让韩国男人成为亚洲最潮型男的发型。2年前,他和一个受过良好教育的专业女性结婚了,对方也来自北朝鲜。他们在首尔住的是同一个社区,一位邻居帮他俩居中牵线。初次约会时,俊相发现他之前曾在图书馆看到她在念书,他很喜欢她就像他一样好学这一点。我参加了这场婚礼,地点是在首尔南部一栋玻璃帷幕办公大楼里的婚宴厅。宾客稀稀落落,因为他和他太太在韩国都没有家人,朋友也不多。\\

在这本书里的所有人当中,最令我讶异的是金赫。他回学校继续学业,读完高中和大学,又从韩国政府附设的统一研究院获得硕士学位。上次交谈时,金赫说他在攻读研究北朝鲜局势的博士学位。他说他未婚,因为现阶段的人生太忙了。在休闲娱乐的方面,他踢足球、上健身房、滑雪。金赫是这本书里最公开的一位人士。他的人生被拍成一部动画短片。他在联合国人权委员会面前为北朝鲜的人权状况做见证。\\

在北朝鲜没有幸存的家人使得他比较大胆。随着时间过去,我发现我在韩国认识的脱北者变得越来越谨慎惶恐。他们担心脱北者的圈子里会有试图举报他们的间谍。他们害怕谈论人权或接受记者访问会招来报复。你可以离开北朝鲜,但永远也没办法完全脱离那份恐惧。\\

\begin{flushright}
	芭芭拉德米克 2015年3月
\end{flushright}


\begin{center}
	{\FA } \\
	全书完
\end{center}
