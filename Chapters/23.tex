\fancyhead[LO]{{\scriptsize {\FA \ }我们最幸福 {\FA } 书中角色}}%奇數頁眉的左邊
\fancyhead[RO]{{\tiny{\textcolor{Gray}{\FA \ }}}\thepage}
\fancyhead[LE]{{\tiny{\textcolor{Gray}{\FA \ }}}\thepage}
\fancyhead[RE]{{\scriptsize {\FA \ }我们最幸福 {\FA } 书中角色}}%偶數頁眉的右邊
\fancyfoot[LE,RO]{}
\fancyfoot[LO,CE]{}
\fancyfoot[CO,RE]{}
\chapter*{书中角色}
\addcontentsline{toc}{chapter}{\hspace{5mm}书中角色}
\vspace{10mm}
\begin{flushright}
	\textcolor{PinYinColor}{\EN \huge{Main\\
	Characters\\
			\ \\}}
\end{flushright}

本书主要内容来自7年来对以下六位脱北者的讲述,附加一些国际机构驻朝、韩工作人员提供的信息及作者本人的亲身经历及见闻。为保护文中涉及的朝鲜人士隐私,均使用化名代称\footnote{中文上方注音为英文原版中的姓名拼写,方便读者与原版对照。}。\\


\begin{description}
	\item[\ruby{宋女士}{\textcolor{PinYinColor}{Mrs. Song}}] 全名\ruby{宋熙锡}{\textcolor{PinYinColor}{Song Hee-suk}},家庭主妇。起初为朝鲜的坚定信仰者,在女儿玉熙强制将其接到中国后,通过在中国的亲身经历中国的富足生活,开始怀疑朝鲜当局的谎言。后主动通过女儿的帮助抵达韩国。到韩国定居后,对以往在朝鲜的生活感慨颇多。虽年近七旬,目前仍同女儿们一起经营Karaoke生意。她每年和依旧住在朝鲜清津的哥哥姐姐联络几次。
		{\footnotesize \begin{description}
			\item[\ruby{长博}{\textcolor{PinYinColor}{Chang-bo}}] 宋女士的丈夫。朝鲜电视台的记者,劳动党党员。1997年因饥荒去世。他的母亲、宋女士的婆婆,已于1996年因饥荒去世。
			\item[\ruby{南玉}{\textcolor{PinYinColor}{Nam-oak}}] 宋女士的儿子、玉熙的弟弟。1998年因饥荒去世。
			\item[\ruby{容熙}{\textcolor{PinYinColor}{Yong-hee}}] 宋女士的小女儿、南玉和玉熙的妹妹。后经过玉熙帮助,携丈夫及孩子一同来到韩国,跟玉熙一起经营Karaoke生意。
		\end{description}}
		
	\item[\ruby{玉熙}{\textcolor{PinYinColor}{Oak-hee}}] 宋女士的大女儿。曾在朝鲜一个建筑公司宣传部门的工作。在不堪忍受家暴后离婚,经历过多次脱北及被遣送回朝鲜,其间还曾同一名中国农民结过婚。最终定居在韩国,经营Karaoke生意。并通过不懈努力,把母亲宋熙锡、妹妹容熙全家及自己的女儿陆续接到韩国。
		{\footnotesize \begin{description}
			\item[\ruby{永洙}{\textcolor{PinYinColor}{Yong-su}}] 玉熙的前夫。因经常家暴及出轨,玉熙与其离婚。目前他们的儿子仍跟永洙一起生活在朝鲜。
		\end{description}}

	\item[\ruby{美兰}{\textcolor{PinYinColor}{Mi-ran}}] 韩国战俘泰宇的女儿。因父亲在朝鲜“出身不佳”,难以提升社会地位,因怕拖累俊相而迟迟无法公开恋情。在朝鲜时是一名幼儿园的教师。在1998年脱北之后,由于与韩国亲戚的纽带关系,因此生活较大部分脱北者顺利。后与韩国男子结婚并育有两个孩子,生活在韩国首尔江南区。
		{\footnotesize \begin{description}
				\item[\ruby{泰宇}{\textcolor{PinYinColor}{Tae-Woo}}] 美兰的父亲。1932年生于现韩国忠清南道,朝鲜一个高岭土矿的木工,1997年因饥荒去世。
				\item[\ruby{美熙}{\textcolor{PinYinColor}{Mi-hee}}] 美兰的大姐。因美兰等亲属的叛逃,被强被朝鲜政府制离婚并收押在劳动营长期服刑。估计在1999年严重的食物短缺时期已去世。
				\item[\ruby{美淑}{\textcolor{PinYinColor}{Mi-sook}}] 美兰的二姐。与大姐美熙一样,因美兰等亲属的叛逃被朝鲜政府强制离婚并收押在劳动营长期服刑。估计在1999年严重的食物短缺时期已去世。
				\item[\ruby{昭熙}{\textcolor{PinYinColor}{So-hee}}] 美兰的三姐。1998年同美兰及母亲一起经中国到达韩国首尔。
				\item[\ruby{锡柱}{\textcolor{PinYinColor}{Sok-ju}}] 美兰的弟弟。1998年同美兰及母亲一起经中国到达韩国首尔。后经过不懈努力被澳大利亚一所大学录取。
		\end{description}}
	
	\item[\ruby{俊相}{\textcolor{PinYinColor}{Jun-sang}}] 朝鲜的大学生。日本朝鲜族后代,美兰在朝鲜时的男朋友。因在平壤上大学过程中与社会接触面不断增大,在美兰脱北前已经有想法要离开朝鲜。当得知米兰脱北后,辞去工作,经过多年准备最终借道中国到达韩国首尔并再次见到了美兰。多年后在韩国跟一位同样是来自朝鲜的姑娘结婚。\\
	
	\item[\ruby{金赫}{\textcolor{PinYinColor}{Kim Hyuck}}] 生于一个坚定的共产主义者家庭,父亲是劳动党党员。母亲去世后,父亲再婚。因金赫兄弟跟继母不合被父亲送到孤儿院。之后成为流浪的燕子。2001年经中国及蒙古国后到达韩国,经过种种努力考入首尔一所大学。金赫是这本书里最公开的一位人士,他的人生被拍成一部动画短片。他在联合国人权委员会面前为北朝鲜的人权状况做见证。
		{\footnotesize \begin{description}
			\item[\ruby{金哲}{\textcolor{PinYinColor}{Kim Cheol}}] 金赫的哥哥。因大金赫3岁,年满后先于金赫离开孤儿院,至今下落不明。
		\end{description}}
	
	\item[\ruby{金医生}{\textcolor{PinYinColor}{Dr. Kim}}] 全名\ruby{金智恩}{\textcolor{PinYinColor}{Kim Ji-eun}},医生。28岁时与前夫离婚。她的父亲在中国有亲戚,父亲去世前留下了中国亲戚的联系方式。当金医生发现劳动党怀疑她因为离婚及父母均已过世,可能会叛逃后,因失望毅然于1999年前往中国通过那里亲戚等人的帮助最终去往韩国。后来她在东首尔开了一家诊所,收入足以让她付钱给掮客,把她的儿子从北朝鲜弄出来。她的丈夫因癌症在朝鲜去世。\\
\end{description}