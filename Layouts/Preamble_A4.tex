\documentclass[oneside,openright,headings=optiontohead]{scrbook}
\renewcommand{\baselinestretch}{1.3}  %行間距倍率
\columnsep 7mm
%\renewcommand\thepage{}


\usepackage[
a4paper=true,
%CJKbookmarks,
unicode=true,
bookmarksnumbered,
bookmarksopen,
hyperfigures=true,
hyperindex=true,
pdfpagelayout = SinglePage,
%pdfpagelayout = TwoPageRight,
pdfpagelabels = true,
pdfstartview = FitV,
colorlinks,
pdfborder=001,
linkcolor=black,
anchorcolor=black,
citecolor=black,
pdftitle={Nothing to Envy},
pdfauthor={Barbara Demick},
pdfsubject={我们最幸福},
pdfkeywords={本书是基于7年来对脱北者的访谈。出于保护那些至今仍然生活在北朝鲜的人们,书中我都采用了化名。所有的对话都是取自一名或多名当事人的描述。我也尽我所能将所听到的故事同公开报导的事件进行印证。书中,对于我个人无法亲自参观地点的描述,来自于脱北者的口述、照片或者影像资料。北朝鲜在很多方面,迄今为止,外界仍然不得而知。因而,我也不能保证我所听到的都是事实真相。我所希望的就是,有朝一日,北朝鲜变得开放之后,我们能够自己判断那里到底发生了什么。},
pdfcreator={https://m-mono.github.io}
]{hyperref}


\usepackage{graphics,graphicx,pdfpages}
\usepackage{caption} %用于取消标题编号 \caption*{abc}

%自动加注拼音
\usepackage{xpinyin}
\xpinyinsetup{format={\color{PinYinColor}}}
%手动加注外语及日语振假名
\usepackage{ruby}
\renewcommand\rubysize{0.4} %匹配 xpinyin 默认标注字体大小
\renewcommand\rubysep{-0.5em} %匹配 xpinyin 默认标注高度

\usepackage{xeCJK}
\usepackage{indentfirst}
\setlength{\parindent}{2.0em}

%正文字体
\setCJKmainfont[Path=Fonts/]{SourceHanSansCN-Normal.otf}
\setCJKsansfont[Path=Fonts/]{SourceHanSansCN-Normal.otf}
\setCJKmonofont[Path=Fonts/]{SourceHanSansCN-Normal.otf}
\setmainfont[Path=Fonts/]{SourceHanSansCN-Normal.otf}
\setsansfont[Path=Fonts/]{SourceHanSansCN-Normal.otf}
\setmonofont[Path=Fonts/]{SourceHanSansCN-Normal.otf}


% 頁面及文字顏色
\usepackage{xcolor}
\definecolor{TEXTColor}{RGB}{50,50,50} % TEXT Color
\definecolor{PinYinColor}{RGB}{130,130,130} % TEXT Color
\definecolor{NOTEXTColor}{RGB}{0,0,0} % No TEXT Color
\definecolor{BGColor}{RGB}{240,240,240} % BG Color


\usepackage{multicol}

\makeindex
\renewcommand{\contentsname}{{我们最幸福}}
\usepackage{fancyhdr} % 設置頁眉頁腳
\pagestyle{fancy}
%\fancyhf{} % 清空當前設置
\renewcommand{\headrulewidth}{0pt}  %頁眉線寬,設為0可以去頁眉線
\renewcommand{\footrulewidth}{0pt}  %頁眉線寬,設為0可以去頁眉線

\usepackage{titletoc}
\dottedcontents{section}[100em]{\bfseries}{100em}{100em} % 去掉目录虚线